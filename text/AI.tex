\section{AI}
\subsection{3 podmínky obecné AI}
\begin{itemize}
    \item Kombinatorická generalizace: Hlavní chybějící požadavek pro obecnou AI a klíčový aspekt vývoje AI. Schopnost skládání nových struktur ze známých stavebních bloků. \uv{Nekonečné využití konečných prostředků}. Noty do skladby. Písmena na slova. Slova do vět.  
    \item Relační zdůvodnění: Základní charakteristika inteligence. Schopnost vidět smysluplné spojení/vzorce mezi objekty (Časové vztahy, prostorové vztahy\dots). 
    \item Relační indukční zaměření: Vybírání co nejvalidnější relace pro vyřešení dané otázky.

\end{itemize}
\subsection{Člověk vs funkce AI}
(tady absolutně nemám tušení, co tím myslí a v přednášce jsem to nikde neviděl a jedinej rozdíl je snad ta kombinatorická generalizace, že AI není schopná vidět jakoukoliv jinou relaci, než tu, na kterou se naučila)

\subsection{Předávání zpráv v grafu (princip)}
\TODO{Jsou asi 3 typy v poslední přednášce, zjistit, jestli potřebuje všechny a chtělo by to asi obrázek + tam někde dělá matici záměn a netuším, jestli to chce jako popsat celé}

Graf $H_t$ předá zprávu na graf $H_{t+1}$. Nejspíš potřebuje jen tu první, protože ta se asi obecně použije podle toho, jak je napsaná ta prezentace, ale má tam 3 typy.
\subsubsection{Average ze sousedů a hodnoty sama sebe}
Hodnoty na jednotlivých uzlech se vypočítají jako jako aritmetický průměr sousedních hodnot a samotného uzlu.
\subsubsection{Average ze sousedů bez hodnoty sama sebe}
Hodnoty na jednotlivých uzlech se vypočítají jako aritmetický průměr sousedních hodnot, ale bez samotného uzlu.
\subsubsection{Suma sousedů}
Hodnoty na jednotlivých uzlech se vypočítají jen jako suma sousedních uzlů.

