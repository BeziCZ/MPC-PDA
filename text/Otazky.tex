\section{Teorie složitosti}
\subsection{Druhy složitosti}
\subsubsection{Polynomiální vs exponenciální složitost}
\begin{itemize}
    \item Polynomiální složitost
    \begin{itemize}
        \item S délkou vstupu (\(n\)) roste čas potřebný k vyřešení lineárně (\(n^x\)). Tudíž čas potřebný k řešení roste propocionálně, ale ne exponenciálně.
        \item Lze je řešit bez potřeby vysoké výpočetní síly.
    \end{itemize}
    \item Exponeciální složitost
    \begin{itemize}
        \item Lze je řešit efektivně bez velkého výpočetního výkonu pouze pro malé vstupy
        \item Čas potřebný pro řešení roste s počtem vstupů \(n\) exponenciálně \(x^n\)
    \end{itemize}
\end{itemize}

\subsubsection{Třídy složistoti}
\begin{itemize}
    \item P
    \item NC
    \item NP
    \item NP-Complete 
    \item NP-Hard
    \item \#P
    \item PSPACE
    \item NL
    \item EXP
    \item EXPSPACE
\end{itemize}

\subsubsection{Třída P}
Spadají do ní problémy, které je možné řešit v polynomiálním čas na deterministickém turingově stroji. Spadají sem například problémy jako je nalezení nekratší
cesty nebo nalezení kostry grafu.
\subsubsection{Třída NC}
Podmožina P, která obsahuje problémy, které se dají řešit v polylogaritmickém paralelním čase na polynomiálním počtu procesorů.\\
Problém se vstupem délky \(n\) je v NC pokud existují konstanty \(k\) a \(c\) takové, že platí \(O(k \cdot log^cn)\) pak je možné vyřešit problém za \(O(log^cn)\)
na \(O(n^k)\) paralelních procesorech.\\
P je pro určení míry efektivity algoritmů při sekvenčním provádění a NC je pro určení míry efektivity při paralelním provádění.

\subsubsection{Třída NP}
Obsahuje problémy, které je možné řešit v polynomiálním čase na nedeterministickém turingově stroji.\\
\subsubsection{Třída NP-Complete}
Podmožina NP, která obsahuje nejtěžší problémy v NP. Problém x je NP-Complete pokud je možné za polynomiální transformovat NP problém y na
problém x. Nebylo pro ně nalezeno polynomiální řešení, ale dokázáno, že řešení existuje. Na deterministickém turingově stroji je možné je řešit pouze za
exponenciální čas. Pokud by bylo nalezeno řešení v polynomiálním čase pak by to znamenalo, že P=NP, protože nalezení řešení pro jeden NP-Complete problém
znamená nalezení řešení pro všechny NP-Complete problémy.\\
Do NP-C spadá například barvení grafů, Knapsack problem(problém batohu), 3-partition problem(rozdělení množiny čísel na trojice se stejným součtem).

\subsubsection{NP-Hard}
Obsahuje problémy, které jsou alespoň tak těžké jako jsou NP-Complete problémy, ale zároveň se neví jestli patří do NP. Problém x je NP-Hard pokud se na něj
redukuje problém y který je ze třídy NP-Complete za polynomiální čas. Existují také problémy, které jsou NP-Hard, ale ne NP-Complete,
například problém zastavení turingova stroje nebo problém splnitelnosti booleovské formule.

