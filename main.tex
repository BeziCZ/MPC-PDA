%%%%%%%%%%%%%%%%%%%%%%
% NASTAVENÍ FEKT.TEX %
%%%%%%%%%%%%%%%%%%%%%%

% Pokud následující řádky zakomentujete, na titulní straně se nezobrazí.

% Nadpis dokumentu (kód předmětu)
\newcommand{\name}{MPC-PDA}
% Podnadpis dokumentu (název předmětu)
%\newcommand{\subname}{}
% Seznam autorů
\newcommand{\authors}{Studenti IBE}
% Seznam korektorů
%\newcommand{\corrections}{}
% Popis dokumentu
\newcommand{\docdesc}{Příprava na zkoušku 2023}
% Zařazení dokumentu (studijní program)
\newcommand{\docgroup}{Informační bezpečnost, FEKT VUT}
% Odkaz
\newcommand{\docurl}{https://github.com/BeziCZ/MPC-PDA}

% Přepsáním argumentu na 'true' zapnete balíček 'minted' pro sázení kódu.
% Pro jeho použití lokálně musíte mít v systému dostupný Python 3, python
% knihovnu 'minted' a PDFLaTeX musíte spouštět s argumentem '-shell-escape'.
% Místo něj můžete použít prostředí 'lstlisting'.
\newcommand{\docminted}{false}


%%%%%%%%%%%%%%%%%%%%
% OBECNÉ NASTAVENÍ %
%%%%%%%%%%%%%%%%%%%%

\newcommand{\fekttexversion}{2.0}

\documentclass[
    % Velikost základního písma je 12 bodů
    12pt,
    % Formát papíru je A4
    a4paper,
    % Oboustranný tisk
    twoside,
    % Záložky a metainformace ve výsledném PDF budou v kódování unicode
    unicode,
]{article}

% Kódování zdrojových souborů
\usepackage[utf8]{inputenc}
% Kódování výstupního souboru
\usepackage[T1]{fontenc}
% Podpora češtiny
\usepackage[czech]{babel}

% Geometrie stránky
\usepackage[
    % Horní a dolní okraj
    tmargin=25mm,
    bmargin=25mm,
    % Vnitřní a vnější okraj
    lmargin=30mm,
    rmargin=20mm,
    % Velikost zápatí
    footskip=17mm,
    % Vypnutí záhlaví
    nohead,
]{geometry}

% Zajištění kopírovatelnosti a prohledávanosti vytvořených PDF
\usepackage{cmap}
% Podmínky (pro použití v titulní straně)
\usepackage{ifthen}

%%%%%%%%%%%%%%%
% FORMÁTOVÁNÍ %
%%%%%%%%%%%%%%%

% Nastavení stylu nadpisů
\usepackage{sectsty}
% Formátování obsahů
\usepackage{tocloft}
\setcounter{tocdepth}{1}
% Odstranění mezer mezi řádky v seznamech
\usepackage{enumitem}
\setlist{nosep}
\setitemize{leftmargin=1em}
\setenumerate{leftmargin=1.5em}
\renewcommand{\labelitemi}{--}
\renewcommand{\labelitemii}{$\circ$}
\renewcommand{\labelitemiii}{$\cdot$}
\renewcommand{\labelitemiv}{--}
% Sázení správných uvozovek pomocí '\enquote{}'
\usepackage{csquotes}
% Vynucení umístění poznámek pod čarou vespod stránky
\usepackage[bottom]{footmisc}
% Automatické zarovnání textu k předcházení vdov a parchantů
\usepackage[defaultlines=3,all=true]{nowidow}
% Zalomení části textu pokud není na současné stránce dost místa
\usepackage{needspace}
% Nastavení řádkování
\usepackage{setspace}
\onehalfspacing
% Změna odsazení odstavců
\setlength{\parskip}{1em}
\setlength{\parindent}{0em}

% Bezpatkové sázení nadpisů
\allsectionsfont{\sffamily}
% Změna formátování nadpisu a podnadpisů v Obsahu
\renewcommand{\cfttoctitlefont}{\Large\bfseries\sffamily}
\renewcommand{\cftsubsecdotsep}{\cftdotsep}

% Použití moderní/aktualizované sady písem
\usepackage{lmodern}

%%%%%%%%%%%
% NADPISY %
%%%%%%%%%%%

\usepackage{titlesec}

\titlespacing*{\section}{0pt}{10pt}{-0.2\baselineskip}
\titlespacing*{\subsection}{0pt}{0.2\baselineskip}{-0.2\baselineskip}
\titlespacing*{\subsubsection}{0pt}{0.2\baselineskip}{-0.2\baselineskip}
\titlespacing*{\paragraph}{0pt}{0pt}{1em}

%%%%%%%%%%
% ODKAZY %
%%%%%%%%%%

% Tvorba hypertextových odkazů
\usepackage[
    breaklinks=true,
    hypertexnames=false,
]{hyperref}
% Nastavení barvení odkazů
\hypersetup{
    colorlinks,
    citecolor=black,
    filecolor=black,
    linkcolor=black,
    urlcolor=blue
}

%%%%%%%%%%%%%%%%%%%%%%%%%%%
% OBRÁZKY, GRAFY, TABULKY %
%%%%%%%%%%%%%%%%%%%%%%%%%%%

% Vkládání obrázků
\usepackage{graphicx}
\usepackage{subfig}
% Nastavení popisů obrázků, výpisů a tabulek
\usepackage{caption}
\captionsetup{justification=centering}
% Grafy a vektorové obrázky
\usepackage{tikz}
\usetikzlibrary{shapes,arrows}
% Složitější tabulky
\usepackage{tabularx}
\usepackage{multicol}

% Sázení osamocených float prostředí v horní části stránky
\makeatletter
\setlength{\@fptop}{0pt plus 10pt minus 0pt}
\makeatother

% Vynucení vypsání floating prostředí pomocí \FloatBarrier
\usepackage{placeins}

% Rámečky
\usepackage{mdframed}

%%%%%%%%%%%%%%
% MATEMATIKA %
%%%%%%%%%%%%%%

% Sázení matematiky a matematických symbolů ('\mathbb{}')
\usepackage{amsmath}
\usepackage{amssymb}
% Sázení fyzikálních veličin
\usepackage{siunitx}

%%%%%%%%%%%%%%%%%
% ZDROJOVÉ KÓDY %
%%%%%%%%%%%%%%%%%

% Sazba zdrojových kódů
\usepackage[formats]{listings}
% Přepnutí prostředí 'code' do režimu výpisu kódu
\newenvironment{code}{\captionsetup{type=listing}}{}

\lstset{
    basicstyle=\small\ttfamily,
    numbers=left,
    numberstyle=\tiny,
    tabsize=4,
    columns=fixed,
    showstringspaces=false,
    showtabs=false,
    keepspaces,
}

% Balíček 'minted' budeme používat pouze pokud je jeho hodnota nastavena na 'true'
\providecommand{\docminted}{false}
\ifthenelse{\equal{\docminted}{true}}
{
    % Sazba zdrojových kódů
    \usepackage[newfloat]{minted}
    % Nastavení barev 'minted' kódů
    \usemintedstyle{pastie}
}
{
    % \docminted není 'true', nic neprovádíme
    % Pokud je v dokumentu 'minted' prostředí, dokument se nepodaří přeložit.
}

%%%%%%%%%%%%%%%%%%%
% VLASTNÍ PŘÍKAZY %
%%%%%%%%%%%%%%%%%%%

\newcounter{todo}
\newcommand{\TODO}[1]{%
    \addtocounter{todo}{1}%
    \textcolor{red}{%
    \textbf{\sffamily\small{TODO \thetodo}%
    \ifthenelse{\equal{#1}{}}{}{:}%
    } %
    #1%
    }%
}

%%%%%%%%%%%
% TITULKA %
%%%%%%%%%%%

\newcommand{\titulka}{
    \vspace*{2em}
    \begin{center}
        \ifthenelse{\isundefined{\name}}{}{{\Huge \bfseries \name{}}}

        \ifthenelse{\isundefined{\subname}}{}{{\huge \bfseries \subname{}}}

        \vspace*{2em}

        \ifthenelse{\isundefined{\docdesc}}{}{{\Large \docdesc}}

        \vspace*{1em}

        \ifthenelse{\isundefined{\docgroup}}{}{\docgroup}

        \ifthenelse{\isundefined{\docurl}}{}{\url{\docurl}}
    \end{center}

    \vfill

    \ifthenelse{\isundefined{\authors}}{}{\authors{}}
    \ifthenelse{\isundefined{\corrections}}{}{\\\small (korektury \corrections{})}

    {}{\small \today}
    \\{\small FEKT.tex \fekttexversion{}}

    \thispagestyle{empty}
    \newpage
}

%%%%%%%%%%%%
% DOKUMENT %
%%%%%%%%%%%%

\begin{document}

\titulka{}

\tableofcontents
\thispagestyle{empty}

\setcounter{page}{0}

\section{Teorie složitosti}
\subsection{Druhy složitosti}
\subsubsection{Polynomiální vs exponenciální složitost}
\begin{itemize}
    \item Polynomiální složitost
    \begin{itemize}
        \item S délkou vstupu (\(n\)) roste čas potřebný k vyřešení lineárně (\(n^x\)). Tudíž čas potřebný k řešení roste propocionálně, ale ne exponenciálně.
        \item Lze je řešit bez potřeby vysoké výpočetní síly.
    \end{itemize}
    \item Exponeciální složitost
    \begin{itemize}
        \item Lze je řešit efektivně bez velkého výpočetního výkonu pouze pro malé vstupy
        \item Čas potřebný pro řešení roste s počtem vstupů \(n\) exponenciálně \(x^n\)
    \end{itemize}
\end{itemize}

\subsubsection{Třídy složistoti}
\begin{itemize}
    \item P
    \item NC
    \item NP
    \item NP-Complete 
    \item NP-Hard
    \item \#P
    \item PSPACE
    \item NL
    \item EXP
    \item EXPSPACE
\end{itemize}

\subsubsection{Třída P}
Spadají do ní problémy, které je možné řešit v polynomiálním čas na deterministickém turingově stroji. Spadají sem například problémy jako je nalezení nekratší
cesty nebo nalezení kostry grafu.
\subsubsection{Třída NC}
Podmožina P, která obsahuje problémy, které se dají řešit v polylogaritmickém paralelním čase na polynomiálním počtu procesorů.\\
Problém se vstupem délky \(n\) je v NC pokud existují konstanty \(k\) a \(c\) takové, že platí \(O(k \cdot log^cn)\) pak je možné vyřešit problém za \(O(log^cn)\)
na \(O(n^k)\) paralelních procesorech.\\
P je pro určení míry efektivity algoritmů při sekvenčním provádění a NC je pro určení míry efektivity při paralelním provádění.

\subsubsection{Třída NP}
Obsahuje problémy, které je možné řešit v polynomiálním čase na nedeterministickém turingově stroji.\\
\subsubsection{Třída NP-Complete}
Podmožina NP, která obsahuje nejtěžší problémy v NP. Problém x je NP-Complete pokud je možné za polynomiální transformovat NP problém y na
problém x. Nebylo pro ně nalezeno polynomiální řešení, ale dokázáno, že řešení existuje. Na deterministickém turingově stroji je možné je řešit pouze za
exponenciální čas. Pokud by bylo nalezeno řešení v polynomiálním čase pak by to znamenalo, že P=NP, protože nalezení řešení pro jeden NP-Complete problém
znamená nalezení řešení pro všechny NP-Complete problémy.\\
Do NP-C spadá například barvení grafů, Knapsack problem(problém batohu), 3-partition problem(rozdělení množiny čísel na trojice se stejným součtem).

\subsubsection{NP-Hard}
Obsahuje problémy, které jsou alespoň tak těžké jako jsou NP-Complete problémy, ale zároveň se neví jestli patří do NP. Problém x je NP-Hard pokud se na něj
redukuje problém y který je ze třídy NP-Complete za polynomiální čas. Existují také problémy, které jsou NP-Hard, ale ne NP-Complete,
například problém zastavení turingova stroje nebo problém splnitelnosti booleovské formule.



\end{document}
